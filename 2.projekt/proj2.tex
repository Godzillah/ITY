\documentclass[11pt,a4paper,twocolumn]{article}
\usepackage[czech]{babel}
\usepackage[utf8x]{inputenc}
\usepackage[left=1.5cm,text={18cm, 25cm},top=2.5cm,a4paper]{geometry}
\usepackage[IL2]{fontenc}
\usepackage{times}
\usepackage{amsthm}
\usepackage{amsmath}
\usepackage{amsfonts}
\usepackage{footmisc}

\theoremstyle{definition}
\newtheorem{definice}{Definice}
\theoremstyle{plain}
\newtheorem{algoritmus}[definice]{Algoritmus}
\newtheorem{veta}{Věta}

\begin{document}


\begin{titlepage}
\begin{center}
\textsc{\Huge Fakulta informačních technologií\\
Vysoké učení technické v~Brně}\\
\vspace{\stretch{0.382}}
\LARGE Typografie a~publikování\,--\,2. projekt\\
Sazba dokumentů a~matematických výrazů\\
\vspace{\stretch{0.618}}
\Large 2018 \hfill         Maroš Orsák (xorsak02) \newpage
\end{center}
\end{titlepage}



\section*{Úvod}

V~této úloze si vyzkoušíme sazbu titulní strany,~matematických vzorců,~prostředí a~dalších textových struktur obvyklých pro technicky zaměřené texty (například rovnice (\ref{rovnice}) nebo definice \ref{definice} na straně \pageref{definice}). Rovněž si vyzkoušíme používání odkazů \verb|\ref| a~\verb|\pageref|.

Na titulní straně je využito sázení nadpisu podle optického středu s~využitím zlatého řezu. tento postup byl probírán na přednášce. Dále je použito odřádkování se zadanou relativní velikostí 0.4em a~0.3em.

\section{Matematický text}

Nejprve se podíváme na sázení matematických symbolů a~výrazů v~plynulém textu včetně sazby definic a~vět s~využitím balíku \verb|amsthm|.Rovněž použijeme poznámku pod čarou s~použitím příkazu \verb|\footnote|. Někdy je vhodné použít konstrukci \verb|${}$|, která říká, že matematický text nemá být zalomen.

\begin{definice}\label{definice}

Turingův stroj \textit{(TS) je definován jako šestice tvaru $M = (Q ,\Sigma ,\Gamma , \delta , q_0 , q_F)$ , kde:}

\begin{itemize}
	
	\item $Q$ \textit{je konečná množina} vnítřních (řídicích) stavů,
	\item $\Sigma$\textit{ je konečná množina symbolů nazývaná} vstupní abeceda , $\Delta \notin \Sigma$,
	\item $\Gamma$\textit{ je konečná množina symbolů, $\Sigma \subset \Gamma , \Delta  \in \Gamma$, nazývaná} pásková abeceda,
	
	\item {\textit{$\delta:(Q\setminus\{q_F\})\times\Gamma\rightarrow Q\times(\Gamma\cup\{L,R\}),\text{kde }L,R\notin\Gamma\text{ ,je parciální přechodová funkce},$}}

 	\item \textit{$q_0$ je} počáteční stav, $q_0 \in Q$ a
 	
 	\item \textit{$q_F$ je} koncový stav, $q_F \in Q$.
	   
\end{itemize}

Symbol $\Delta$ značí tzv. \textit{blank} (prázdny symbol), který se vyskytuje na místech pásky, která nebyla ještě použita (může ale být na pásku zapsán i~později).

\textit{Konfigurace pásky} se skladá z~nekonečného řetězce, který reprezentuje obsah pásky a~pozice hlavy na tomto řetězci.Jedná se o~prvek množiny$\{\gamma\Delta^\omega|\gamma\in\Gamma^*\}\times\mathbb{N}.$\footnote{Pro livolnou abecedu $\Sigma$ a~$\Sigma^\omega \text{ množina všech } $\textit{nekonečných} řetězců nad $\Sigma$, tj. nekonečných posloupností symbolů ze $\Sigma$.Pro připomenutí: $\Sigma^*$ je množina všech \textit{konečných} řetězců nad $\Sigma$.}
\textit{Konfiguraci pásky} obvykle zapisujeme jako $\Delta xyz\underline{z}x\Delta...$(podtržení značí pozici hlavy).\textit{ Konfigurace stroje} je pak dána stavem řízení a~konfigurací pásky. Formálně se jedná o~prvek množiny $Q\times\{\gamma\Delta^\omega|\gamma\in\Gamma^*\}\times\mathbb{N}.$
\end{definice}\label{definice}


\subsection{Podsekce obsahujíci větu a~odkaz}

\begin{definice}
Řetězec $w$ nad abecedou $\Sigma$ je přijat TS \textit{M jestliže M je při aktivaci z~počateční konfigurace pásky $\underline{\Delta}w\Delta$... a~počatečního stavu $q_0$ zastaví přechodem do koncového stavu $q_F$ , tj. $(q_0,\Delta w\Delta^\omega,0){{\vdash}^M_*} (q_F,\gamma,n)$ pro nějaké $\gamma\in\Gamma^*$ a~$n\in\mathbb{N}$.}

\textit{Množinu $L(M) = \{w|w$je přijat TS $M\}\subseteq\Sigma^*$ nazýváme jazyk přijímaný TS $M$.}
\vspace{2mm}

Nyní si vyskoušíme sazbu vět a~důkazů opět s~použitím balíku \verb|amsthm|.

\begin{veta}
\textit{Třída jazyků, které jsou příjímány TS, odpovídá rekurzivně vyčíslitelným jazykům.}
\end{veta}

\begin{proof}
Důkaz. V~důkaze vyjdeme z~Definice 1 2.
\end{proof}

\end{definice}

\section{Rovnice a~odkazy}
Složitější matematické formulace sázíme mimo plynulý text.Lze umístit několil výrazů na jeden řádek, ale pak je třeba tyto vhodně oddělit, například příkazem \verb|\quad|.

\begin{equation}
\nonumber
\sqrt[i]{x_i^3} \quad \text{kde } x_i \text{ je } i\text{-té sudé číslo} \quad y_i^{2^y_i} \neq y_i^{{y_i}^{y_i}} 
\end{equation}

V~rovnici ($1$) jsou využity tři typy závorek s~různou explicitně definovanou velikostí.

\setcounter{equation}{0}

\begin{eqnarray}\label{rovnice}
 x =\Bigg\{\bigg(\big[a+b]* c\bigg)^d\oplus 1\Bigg\}   \\
 y =\lim_{x\to\infty} \frac{\sin^2 
x+\cos^2 x}{\frac{1}{\log_{10}{x}}} \nonumber
\end{eqnarray}

V~této větě vidíme, jak vypadá implicitní vysázení limity $\lim_{n\to\infty} f(n)$v normálním odstavci textu. Podobně je to i~s~dalšími symboly jako $\sum_{i=1}^{n} 2^i$ či $\bigcup_{A\in B}$A. V~případě vzorců $\lim\limits_{x \to \infty}{f(n)}$ a~$\sum_{i=1}^{n} 2^i$ jsme si vynutili méně úspornou sazbu příkazem \verb|\limits|.

\begin{eqnarray}
\int\limits_{a}^{b}f(x)\,\mathrm{d}x & = & -\int_b^ag(x)\,\mathrm{d}x\\
\overline{\overline{A\vee B}} & \Leftrightarrow & \overline{\overline{A}\wedge \overline{B}}\ \end{eqnarray}

\section{Matice}

Pro sázení matic se velmi často používá prostředí \verb|array| a~závorky(\verb|\left|,\verb|\right|).


$$ \left(
\begin{array}{c l l}
a+b & \widehat{\xi +\omega} & \hat{\pi}  \\
\vec{a} & \overleftrightarrow{AC} & \beta  
\end{array}\right) 
= 1 \Longleftrightarrow
\mathbb{Q} = \mathbb{R}
$$


% DETERMINANT...
$
{A = \left\|
 \begin{matrix}
  a_{11} & a_{12} & \cdots & a_{1_n} \\
  a_{21} & a_{22} & \cdots & a_{2_n} \\
  \vdots  & \vdots  & \ddots & \vdots  \\
  a_{m1} & a_{m2} & \cdots & a_{mn} 
 \end{matrix}\right\| =  
  \left|
\begin{array}{c l}
t & u \\
v & w
\end{array}\right| =tw-uv}\\\\
$

Prostředí \verb|array| lze úspešně využít i~jinde.

$$
\binom{n}{k} = \begin{cases}
\ \frac{n!}{k!(n - k)!} & \text{pro } 0 \leq k~\leq n\\
\ 0 & \text{pro } k~< 0 \text{ nebo } k~> n
\end{cases}
$$


\section{Závěrem}

V~případě, že budete potřebovat vyjádřit matematickou konstrukci nebo symbol a~nebude se Vám dařit jej nalézt~v~samotném \LaTeX~u, doporučuji prostudovat možnosti balíku maker \AmS-\LaTeX.




\end{document}



