\documentclass[11pt,a4paper,titlepage]{article}
\usepackage[left=2cm,text={17cm,24cm},top=3cm]{geometry}
\usepackage[T1]{fontenc}
\usepackage[czech]{babel}
\usepackage[utf8]{inputenc}
\usepackage{url}
\usepackage[resetlabels]{multibib}

\bibliographystyle{czplain}

\begin{document}

\begin{titlepage}
 \begin{center}
    \LARGE\textsc{Vysoké učení technické v~Brně}\\
    \Large\textsc{Fakulta informačních technologií}\\

    \vspace{\stretch{0.382}}
    \LARGE{Typografie a publikování\,--\,4.\,projekt}\\
    \Huge{Typografia}\\
    \vspace{\stretch{0.618}}
 \end{center}
    {\Large\today \hfill Maroš Orsák }
\end{titlepage}

\section{Úvod}

V~klasickej egyptčine existovalo 750 hieroglyfických znakov, ale počet znakov sa v~rôznych obdobiach menil. Hieroglyfy sú malé obrázky, ktoré vyjadrujú jednak priamo patričný predmet, na ktorý sa podobajú, ale používajú sa aj na vyjadrenie celkom odlišných vecí, často ťažko znázorniteľných, ktorých mená znejú náhodou podobne. Samohlásky sa v~hieroglyfickom písme vynechávali (nezapisovali). \cite{Wiki_Hieroglyfy}

Na svete existuje veľmi málo kníh, ktoré rozoberajú práve touto témou. Za zmienku stojí napríklad časopis. \cite{Hieroglyfy_Casopis} 



\section{Hieroglyfy}
V~klasickej egyptčine existovalo 750 hieroglyfických znakov \cite{Anglicka_Kniha}, ale počet znakov sa v~rôznych obdobiach menil. Hieroglyfy sú malé obrázky, ktoré vyjadrujú jednak priamo patričný predmet, na ktorý sa podobajú, ale používajú sa aj na vyjadrenie celkom odlišných vecí \cite{Wiki_Egyptske_Pismo}, často ťažko znázorniteľných, ktorých mená znejú náhodou podobne. Samohlásky sa v~hieroglyfickom písme vynechávali (nezapisovali) \cite{Kniha}. Napríklad obrázok kravy označuje kravu, ale aj bohyňu Hathor. Rovnaký znak sa používal aj pre rozličné slová, ktoré rovnako zneli (napr. v~slovenčine „jež“ a „ješ“). Hieroglyfy boli často do detailov vypracované a kolorované. Písali sa viacerými smermi, najčastejšie sprava doľava, pričom piktogramy boli obrátené čelom k~začiatku riadku. Písalo sa atramentom a dutým papyrusovým štetcom \cite{citanie_hieroglyfov}.

\section{Rozlúštenie hieroglyfov}
Najvýznamnejšie podiel na rozlúštenie hieroglyfov majú Thomas Young \cite{Thomas_Description} a Jean-François Champollion \cite{Jean_Description} (1790-1832) \cite{bc_praca_1}. V~roku 1799 pri Napoleonovom ťaženie do Egypta bol urobený nález Rosettskej dosky s~nápisom v~troch typoch písma (v~hieroglyfickom, démotickom a gréckom písme). Táto doska poskytla dostatočný kritický materiál, ktorý Champollionovi umožnil preniknúť v~20. rokoch 19. storočia do reči hieroglyfov. Rozlúštenie tohto písma potom raz navždy vyvrátilo názory, že sa jedná o~písmo idiomatické, čiže obrázkové. Niektoré z~Champolionových poznatkov boli síce časom nahradené novšími poznatky o~tomto písme a jazyku, avšak položil základy porozumenia tomuto písme a jeho zásluhy sa tým nijako neumenšujú \cite{bc_praca_2}.

\newpage
\bibliography{reference}
\end{document}

